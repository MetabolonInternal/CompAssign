\documentclass[11pt]{article}

\usepackage[margin=1in]{geometry}
\usepackage{amsmath,amssymb,amsthm}
\usepackage{bm}
\usepackage{hyperref}
\usepackage{float}
\usepackage{graphicx}
\usepackage{booktabs}
\usepackage{subcaption}
\usepackage{verbatim}
\usepackage{longtable}
\usepackage{fontspec}

\setmainfont{Times New Roman}
\setmonofont{Menlo}

\graphicspath{{./}{images/}{images/rt_pymc_multilevel_pooling_report/}{docs/models/}{docs/models/images/}{docs/models/images/rt_pymc_multilevel_pooling_report/}}

\setlength{\parindent}{0pt}
\setlength{\parskip}{0.7\baselineskip}

\title{Report: RT Regression Model Selection (Ridge Pooling vs Hierarchical Ridge vs Lasso Baselines)}
\author{Bioinformatics Team}
\date{\today}

\begin{document}

\maketitle

\section{Executive summary}

We evaluate model \textbf{form} (not feature engineering) for RT regression, focusing on:
\begin{itemize}
  \item \textbf{Accuracy}: global RMSE/MAE on realtest.
  \item \textbf{Uncertainty quality}: empirical coverage of nominal 95\% prediction intervals and their average width.
  \item \textbf{Practicality}: training time and ability to support cold-start predictions.
\end{itemize}

We compare two ridge-style model forms (and two lasso baselines) for lib208 and lib209 under
\textbf{cap100 training} and \textbf{realtest evaluation}:
\begin{itemize}
  \item \textbf{Ridge (supercategory, PyMC)}: a fast collapsed ridge model fit per \texttt{species\_cluster}.
  \item \textbf{Ridge (supercategory, sklearn)}: a per-group ridge baseline that writes coefficient summaries and
  an approximate predictive distribution.
  \item \textbf{Ridge (partial pooling)}: a hierarchical ridge model fit per \texttt{species} with priors pooled
  through \texttt{species\_cluster}.
  \item \textbf{Lasso (supercategory)} and \textbf{Lasso (species)} baselines from Sally's legacy model bundles.
\end{itemize}

\textbf{Key result:} \textbf{Ridge (partial pooling)} improves RMSE and yields much tighter prediction intervals
with coverage close to nominal (0.95), but costs \textbf{hours of training time} (ADVI).
\textbf{Ridge (supercategory, PyMC)} is essentially instantaneous and safe, but is \textbf{over-conservative}
(coverage $\sim$0.99) and produces much wider windows.
\textbf{Ridge (supercategory, sklearn)} matches the same RMSE (ridge-equivalent point predictions) but produces
\textbf{much narrower windows that under-cover} (coverage $\sim$0.92--0.94), which would systematically exclude
true peaks in downstream peak assignment.

\textbf{Recommendation:} Use \textbf{Ridge (partial pooling)} as the primary model form when we can afford offline
training (e.g., periodic retrains), and keep \textbf{Ridge (supercategory, PyMC)} as a fast fallback / debugging baseline.

\section{Introduction}

Downstream peak assignment benefits from RT predictions that are both accurate and accompanied by
well-calibrated uncertainty (for candidate windowing). At the same time, we require a model form that:
\begin{itemize}
  \item generalizes to \textbf{new species not present in training} (cold-start), and
  \item remains stable in the \textbf{long tail} of sparse compound/group history.
\end{itemize}

This report summarizes our current cap100 $\rightarrow$ realtest experiments, compares ridge pooling variants to
two lasso baselines, and provides a recommendation on model form.

\section{Methods}

\subsection{Data and group definitions}

The production RT CSVs contain rows of curated RT observations with run-level covariates
(\texttt{IS*}/\texttt{RS*}/\texttt{ES\_\*}), and identifiers including:
\begin{itemize}
  \item \texttt{species\_cluster}: supercategory identifier (matrix/supercategory),
  \item \texttt{species}: subgroup identifier nested under \texttt{species\_cluster},
  \item \texttt{comp\_id}: compound id.
\end{itemize}

The stage-1 artifact stores coefficient summaries per group
\[
g = (\texttt{group\_id}, \texttt{comp\_id}),
\]
where \texttt{group\_id} is derived from a configured \texttt{group\_col}. In this work:
\begin{itemize}
  \item \texttt{group\_col=species\_cluster} (supercategory ridge),
  \item \texttt{group\_col=species} (hierarchical ridge; nested within supercategory).
\end{itemize}

\subsection{Feature set}

Let $\bm{x}_i$ be the vector of run-level covariates for test row $i$. For pooling comparisons in this report we
use \textbf{linear anchors only} (no poly2) to avoid confounding ``pooling effect'' with ``feature expansion effect''.

\subsection{Models}

All models are linear-in-features at prediction time and differ in grouping and pooling structure.

\subsubsection{Ridge (supercategory, PyMC): collapsed per-\texttt{species\_cluster} ridge}

For each group $g$ (here: $(\texttt{species\_cluster}, \texttt{comp\_id})$) we model RT:
\[
  y_{gi} = b_g + \bm{x}_{gi}^\top \bm{w}_g + \epsilon_{gi}, \quad \epsilon_{gi} \sim \mathcal{N}(0, \sigma_g^2).
\]
With a Gaussian ridge prior $\bm{w}_g \sim \mathcal{N}(\bm{0}, \tau_w^2 \bm{I})$, the posterior mean matches standard
ridge regression and the posterior covariance is available in closed form. This model trains extremely quickly.

\subsubsection{Ridge (supercategory, sklearn): per-group ridge coefficient summaries}

This baseline trains an independent ridge regression for each group
$g=(\texttt{species\_cluster}, \texttt{comp\_id})$ using the same linear run covariates. It writes a production-friendly
artifact of per-group coefficient posterior summaries (\texttt{beta\_hat}, \texttt{beta\_cov}, \texttt{sigma2\_mean})
and uses a Normal approximation to produce per-row prediction intervals. When we match the ridge penalty $\lambda$ to the
PyMC collapsed ridge and use the same evaluation code, the \emph{point predictions} match up to numerical precision, but the
\emph{interval widths and coverage} can differ substantially.

\subsubsection{Ridge (partial pooling): hierarchical priors for \texttt{species} within \texttt{species\_cluster}}

The partial pooling model keeps the subgroup grouping (\texttt{species}) but introduces hierarchical pooling for:
\begin{itemize}
  \item \textbf{intercepts} with compound-aware structure, and
  \item \textbf{slope prior means} (``slope heads'') with supercategory shrinkage.
\end{itemize}

\paragraph{Intercept prior (\texttt{intercept\_prior=comp\_hier\_supercat}).}
For each subgroup $c$ with parent supercategory $s(c)$ and each compound $k$:
\[
  b_{(c,k)} \sim \mathcal{N}\left(t_0 + \mu_c + \alpha_k, \tau_b\right),
\]
with
\[
  \mu_c \sim \mathcal{N}\left(\mu_{s(c)}, \tau_{\mu,s(c)}\right), \quad
  \mu_s \sim \mathcal{N}(0, \tau_{\mu,\mathrm{supercat}}),
  \quad
  \alpha_k \sim \mathcal{N}(0, \tau_{\mathrm{comp}}).
\]

\paragraph{Slope head (\texttt{slope\_head\_mode=cluster\_supercat}).}
We learn a slope mean for each subgroup $\bm{m}_c \in \mathbb{R}^p$ and use it as the prior mean for group slopes
$\bm{w}_{(c,k)}$. We pool slope heads through the supercategory:
\[
  \bm{m}_c \sim \mathcal{N}\left(\bm{m}_{s(c)}, \tau_{w,s(c)}\right), \quad
  \bm{m}_s \sim \mathcal{N}\left(\bm{m}_0, \tau_{w,\mathrm{supercat}}\right).
\]

\paragraph{Inference.}
We use ADVI for scalability. The exported coefficient summaries incorporate learned prior means so pooling shifts
mean predictions as intended.

\subsubsection{Lasso baselines}

We include two legacy eslasso baselines for context:
\begin{itemize}
  \item \textbf{Lasso (supercategory)}: one lasso per \texttt{species\_cluster}.
  \item \textbf{Lasso (species)}: a smaller set of per-matrix models (``local models'') selected from a
  \texttt{species\_raw} mapping; this baseline does \textbf{not} score all rows.
\end{itemize}
Lasso ``intervals'' are window-based (\texttt{pred} $\pm$ \texttt{window}) rather than derived from a probabilistic
predictive distribution. We still report their empirical ``coverage95'' for comparison, but it should not be
interpreted as nominal calibration in the same way as ridge models.

\subsection{Evaluation metrics}

Let $y_i$ be observed RT and $\hat{y}_i$ be the point prediction on realtest (units: minutes).
\begin{itemize}
  \item \textbf{RMSE}: $\sqrt{\frac{1}{n}\sum_i (\hat{y}_i - y_i)^2}$.
  \item \textbf{MAE}: $\frac{1}{n}\sum_i |\hat{y}_i - y_i|$.
  \item \textbf{Coverage@95\%} (ridge models): with predicted standard deviation
  $\hat{\sigma}_i = \sqrt{\sigma^2_g + \bm{x}_i^\top \widehat{\mathrm{Cov}}(\bm{\beta}_g)\bm{x}_i}$, we define the
  nominal 95\% interval as $\hat{y}_i \pm 1.96\,\hat{\sigma}_i$ and compute the fraction of test points inside.
  \item \textbf{Mean interval width} (ridge models): $\frac{1}{n}\sum_i 2\cdot 1.96\cdot \hat{\sigma}_i$.
  \item \textbf{Coverage@95\% / width (lasso)}: computed using the legacy window:
  $|\hat{y}_i - y_i| \le w_i$ with width $2w_i$.
\end{itemize}

We stratify by \textbf{training support bin} of each $(\texttt{group\_id}, \texttt{comp\_id})$ group using bins:
\[
\{\le 1,\; 2,\; 3\text{--}5,\; 6\text{--}10,\; 11\text{--}20,\; 21\text{--}50,\; 51\text{--}100,\; >100\}.
\]

\subsection{Reproducibility}

Ridge experiments were generated by:
\begin{verbatim}
./scripts/run_rt_prod.sh --cap 100 --libs 208,209
./scripts/run_rt_prod_eval.sh --cap 100 --libs 208,209
\end{verbatim}
and plots/tables in this report were regenerated by:
\begin{verbatim}
./scripts/plot_rt_multilevel.sh --cap 100 --libs 208,209
\end{verbatim}
using the run directory pointer in \texttt{output/rt\_prod\_latest.txt}.

\section{Results}

\subsection{Global performance}

Table~\ref{tab:global_metrics} summarizes global metrics on realtest for lib208 and lib209. Lasso baselines do not
score all rows; \textbf{Rows scored} reflects applicability.

\begin{table}[H]
\centering
\caption{Global realtest metrics for cap100-trained models (linear features).}
\label{tab:global_metrics}
\begin{tabular}{llrrrrrr}
\toprule
Lib & Model & RMSE & MAE & Cov95 & Width95 & Rows scored & Train (min) \\
\midrule
208 & Ridge (supercategory, PyMC) & 0.009023 & 0.004673 & 0.987 & 0.068109 & 100.0\% & 0.0 \\
208 & Ridge (supercategory, sklearn) & 0.009023 & 0.004673 & 0.943 & 0.023534 & 100.0\% & 0.0 \\
208 & Ridge (partial pooling) & 0.007877 & 0.003827 & 0.956 & 0.027978 & 100.0\% & 465.0 \\
208 & Lasso (supercategory) & 0.015081 & 0.007956 & 0.954 & 0.058908 & 94.7\% & -- \\
208 & Lasso (species) & 0.011425 & 0.006510 & 0.994 & 0.068647 & 51.2\% & -- \\
\midrule
209 & Ridge (supercategory, PyMC) & 0.008317 & 0.004677 & 0.993 & 0.070829 & 100.0\% & 0.1 \\
209 & Ridge (supercategory, sklearn) & 0.008317 & 0.004677 & 0.920 & 0.021393 & 100.0\% & 0.0 \\
209 & Ridge (partial pooling) & 0.007718 & 0.004298 & 0.956 & 0.027551 & 100.0\% & 860.9 \\
209 & Lasso (supercategory) & 0.009439 & 0.004954 & 0.993 & 0.050886 & 97.8\% & -- \\
209 & Lasso (species) & 0.007741 & 0.004346 & 0.998 & 0.055400 & 82.7\% & -- \\
\bottomrule
\end{tabular}
\end{table}

\paragraph{Interpretation.}
Relative to Ridge (supercategory, PyMC), Ridge (partial pooling) reduces RMSE by 0.00115 min (0.069 s) on lib208 and
0.00060 min (0.036 s) on lib209. It also shrinks average 95\% interval width by about 0.040--0.043 min
(2.4--2.6 s) and moves coverage closer to the nominal 0.95 target.
The sklearn supercategory ridge matches the same RMSE (ridge-equivalent means) but substantially under-covers, which is
undesirable when the prediction interval is used as an RT filter in peak assignment.

\begin{figure}[H]
\centering
\begin{subfigure}[b]{0.49\linewidth}
\includegraphics[width=\linewidth]{lib208_global_comparison_anchor_none_full.png}
\caption{lib208}
\end{subfigure}
\begin{subfigure}[b]{0.49\linewidth}
\includegraphics[width=\linewidth]{lib209_global_comparison_anchor_none_full.png}
\caption{lib209}
\end{subfigure}
\caption{Global comparison across models (cap100 training, realtest evaluation). Panels show RMSE, coverage@95\%,
mean interval width, and training time.}
\label{fig:global_comparison}
\end{figure}

\subsection{Comparison vs sklearn ridge baseline}

The sklearn baseline and the PyMC supercategory model are both ridge regression models over the same run covariates.
When we fix the ridge penalty $\lambda$ to the same value in both implementations, their posterior means (and therefore
their point predictions) match up to numerical precision. This is expected: with a Gaussian ridge prior on slopes and a
flat intercept prior, the posterior mean equals the ridge solution.

However, the \textbf{prediction intervals} differ because they depend on how we estimate (and propagate) predictive
variance. In particular, the sklearn coefficient-summary artifact estimates a separate noise scale (\texttt{sigma2\_mean})
for each $(\texttt{species\_cluster}, \texttt{comp\_id})$ group, which can lead to \textbf{overly optimistic}
intervals in practice. Empirically this shows up as substantially \textbf{narrower windows} but \textbf{lower-than-nominal}
coverage (0.92--0.94). In a peak-assignment setting where we use the interval as a hard RT gate, this under-coverage
translates directly into a higher \emph{miss rate}: the true peak is excluded from consideration more often.

Figure~\ref{fig:sklearn_candidates_global} shows the three candidate ridge variants (PyMC supercategory, sklearn supercategory,
and PyMC partial pooling). Figure~\ref{fig:sklearn_candidates_support} shows the same comparison stratified by training support.

\begin{figure}[H]
\centering
\begin{subfigure}[b]{0.49\linewidth}
\includegraphics[width=\linewidth]{lib208_global_comparison_anchor_none_candidates.png}
\caption{lib208}
\end{subfigure}
\begin{subfigure}[b]{0.49\linewidth}
\includegraphics[width=\linewidth]{lib209_global_comparison_anchor_none_candidates.png}
\caption{lib209}
\end{subfigure}
\caption{Global comparison of ridge candidates only. The sklearn supercategory model achieves narrower windows by
under-covering (coverage below 0.95). The PyMC supercategory model over-covers with wider windows. PyMC partial pooling
is closest to the desired operating point: narrow windows with near-nominal coverage.}
\label{fig:sklearn_candidates_global}
\end{figure}

\begin{figure}[H]
\centering
\begin{subfigure}[b]{0.49\linewidth}
\includegraphics[width=\linewidth]{lib208_by_support_bin_anchor_none_candidates.png}
\caption{lib208}
\end{subfigure}
\begin{subfigure}[b]{0.49\linewidth}
\includegraphics[width=\linewidth]{lib209_by_support_bin_anchor_none_candidates.png}
\caption{lib209}
\end{subfigure}
\caption{Ridge candidate comparison by training support bin. The same pattern holds across the long tail: sklearn tends
to be narrower but under-calibrated; PyMC partial pooling tightens intervals while maintaining near-nominal coverage.}
\label{fig:sklearn_candidates_support}
\end{figure}

\subsection{Performance by training support bin}

Figure~\ref{fig:support_bin} breaks metrics down by training support bin (long tail to head).

\begin{figure}[H]
\centering
\begin{subfigure}[b]{0.49\linewidth}
\includegraphics[width=\linewidth]{lib208_by_support_bin_anchor_none_full.png}
\caption{lib208}
\end{subfigure}
\begin{subfigure}[b]{0.49\linewidth}
\includegraphics[width=\linewidth]{lib209_by_support_bin_anchor_none_full.png}
\caption{lib209}
\end{subfigure}
\caption{Metrics by training support bin for each model. Coverage@95\% is empirical coverage of the nominal 95\%
prediction interval (ridge models) or window interval (lasso baselines).}
\label{fig:support_bin}
\end{figure}

\subsection{Performance by species\_cluster}

Figure~\ref{fig:by_cluster} aggregates metrics by \texttt{species\_cluster}.

\begin{figure}[H]
\centering
\begin{subfigure}[b]{0.49\linewidth}
\includegraphics[width=\linewidth]{lib208_by_species_cluster_anchor_none_full.png}
\caption{lib208}
\end{subfigure}
\begin{subfigure}[b]{0.49\linewidth}
\includegraphics[width=\linewidth]{lib209_by_species_cluster_anchor_none_full.png}
\caption{lib209}
\end{subfigure}
\caption{Metrics by \texttt{species\_cluster} (supercategory). These aggregates reflect within-supercategory
performance stability and highlight where models under/over-estimate uncertainty.}
\label{fig:by_cluster}
\end{figure}

\section{Discussion}

\paragraph{Accuracy vs uncertainty for peak assignment.}
For downstream peak assignment, tighter uncertainty windows (with maintained calibration) directly reduce the set
of candidate peaks per compound, improving precision. Ridge (partial pooling) produces substantially narrower
intervals while keeping coverage close to nominal; Ridge (supercategory, PyMC) is consistently over-conservative with
much wider windows.

\paragraph{Why we do not prefer sklearn intervals (as-is).}
The sklearn ridge baseline produces very narrow intervals, but empirically under-covers the nominal 95\% target.
In a peak assignment pipeline that uses RT intervals as a hard filter, under-coverage becomes a direct failure mode:
the true peak is excluded more often. The PyMC partial pooling model achieves a better operating point: narrow windows
with coverage close to nominal, so it reduces candidate load without sacrificing recall.

\paragraph{Cost of hierarchy.}
The benefit of partial pooling comes at a large training cost: $\sim$7.8 hours (lib208) and $\sim$14.3 hours
(lib209) in these experiments. If we retrain infrequently (e.g., periodic offline retrains), this may be acceptable;
if we need rapid iteration, Ridge (supercategory, PyMC) remains valuable.

\paragraph{Cold-start behaviour.}
Both ridge forms can provide predictions for new species as long as the row can be assigned to an existing
\texttt{species\_cluster}. The hierarchical model additionally defines a principled prior over unseen \texttt{species}
within a supercategory, but production usage would require explicitly sampling or constructing coefficients for
unseen groups (future work).

\paragraph{Lasso baselines.}
Lasso (supercategory) is generally weaker in RMSE than ridge candidates. Lasso (species) can look competitive on
lib209 but scores only a subset of rows (and therefore requires a fallback model). Its interval is a fixed window,
so coverage should be interpreted as ``fraction inside the window'' rather than probabilistic calibration.

\section{Conclusion and recommendation}

Based on current cap100 $\rightarrow$ realtest results, \textbf{Ridge (partial pooling)} is the best overall model
form for peak assignment: it improves RMSE and substantially tightens predictive windows with near-nominal
coverage. The main downside is training time (hours).

We recommend:
\begin{itemize}
  \item \textbf{Primary}: Ridge (partial pooling) for production-quality coefficients and calibrated intervals.
  \item \textbf{Fallback / fast iteration}: Ridge (supercategory, PyMC) as a fast, stable baseline and safety net.
  \item \textbf{sklearn ridge}: useful as a fast baseline and for debugging; avoid using its intervals for gating
  unless we explicitly calibrate them to hit nominal coverage.
\end{itemize}

\section{Future extensions}

\begin{itemize}
  \item \textbf{Peak-assignment integration:} use the predictive distribution as a likelihood term (not only a hard gate)
  and tune operating points explicitly in terms of candidate count vs miss rate.
  \item \textbf{Calibration as a first-class step:} fit a small global (or per-supercategory) predictive-scale calibration
  on a held-out validation set to hit nominal coverage while minimizing window width.
  \item \textbf{Active learning loop closure:} choose which species/compounds to curate next based on assignment ambiguity
  (high candidate count), long-tail support, and calibration failures; retrain periodically and measure end-to-end impact.
  \item \textbf{Instrument transfer:} add an explicit transfer layer (e.g., per-instrument offset/scale, or hierarchical
  run-drift components) so new instruments can adapt using limited anchor data without losing calibration.
  \item \textbf{Cold-start matrices and unseen chemicals:} extend intercept priors to chemistry-informed backoff (e.g.,
  embedding or class-conditioned priors) so we can produce sensible means and honest uncertainty for unseen compounds and/or
  new matrices.
  \item \textbf{Training speed and operational simplicity:} reduce ADVI wall time (structured VI, fewer parameters, better
  initialization) and provide clean production scripts with sensible hardcoded defaults for reproducibility.
\end{itemize}

\end{document}
